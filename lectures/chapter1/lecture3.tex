\documentclass[main.tex]{subfiles}
\begin{document}
\section{Базис и размерность векторного пространства}


\newtheorem*{Definition1}{Определение}
\begin{Definition1}
Бесконечная система векторов называется \textbf{линейно незасисимой}, если линейно независима её любая конечная подсистема, и \textbf{линейно зависимой}, если существует линейно зависимая подсистема.
\end{Definition1}


\newtheorem*{Definition2}{Определение}
\begin{Definition2}
Векторное пространство называется \textbf{бесконечномерным}, если в нём существует бесконечная линейно независимая система векторов, и \textbf{конечномерным}, если в ней все линейно независимые системы конечны.
\end{Definition2}


\newtheorem*{Definition3}{Определение}
\begin{Definition3}
\obeylines
Система векторов B $-$ \textbf{базис} векторного пространства $V$, если:
\hfill
1. $B$ $- $ линейно независимая система векторов
2. $L(B) = V$,  то есть любой вектор из $V$ линейно выражается через  систему $B$.
\end{Definition3}


\newtheorem*{t1}{Теорема}
\begin{t1}
В конечномерном векторном пространстве линейно независимая система векторов 
либо является базисом, либо может быть дополнена до базиса.
\begin{proof}
Так как рассматриваемое пространство конечно, то все линейно независимые системы векторов в нём тоже конечны.
Пусть $A = (a_{1},\dots,a_{n})$ $-$ линейно независимая система векторов. Если система векторов $A$ не является базисом, то
существует вектор $b$, который не выражается через $A$. Рассмотрим систему векторов $A_{1} = (a_{1},\dots,a_{n},b).$ 
Если $A_{1}$ линейно зависима, то по свойству линейной независимости вектор $b$ линейно выражается через остальные векторы системы $A_{1},$
т.е через систему $A$. Противоречие $\Rightarrow$ $A_{1}$ линейно независимая система векторов. Таким образом, если линейно независимая система векторов 
не является базисом, то существует вектор, присоединив который к системе $A$ получим линейно независимую систему векторов. Так как рассматриваемое пространство
конечномерно, то бесконечно присоединять векторы к системе сохраняя линейную независимость нельзя $\Rightarrow$ через конечное число шагов процесс остановится 
и мы получим линейно независимую систему векторов через которую выражается любой вектор пространства $V$, т.е базис. Причём система $A$ будет его подсистемой.
\end{proof}
\end{t1}


\newtheorem*{result}{Следствие}
\begin{result}
В ненулевом конечномерном векторном пространстве существует конечный базис.
\begin{proof}
Если вектор $a$ ненулевой, то система $A$ линейно независима $\Rightarrow$ либо базис, либо может быть дополнена до базиса.
\end{proof}
\end{result}


\newtheorem*{t2}{Теорема}
\begin{t2}
Все базисы ненулевого конечномерного векторного пространства состоят из одного и того же количества векторов.
\begin{proof}
Пусть $A = (a_{1},\dots,a_{k})$, $B = (b_{1},\dots,b_{n})$ $-$ базисы $V$ и пусть $k>n$.
Рассмотрим $B_{1} = (a_{1}, b_{1},\dots,b_{n})$. Так как $B$ базис, то вектор $a_{1}$ линейно выражается через $B$, т.е линейной выражается через остальные векторы системы $B_{1}$, тогда $L(B_{1}) = L(B) = V$. С другой стороны так как $a_{1}$ линейно выражается через остальные векторы $B_{1}$, то система $B_{1}$ линейно зависимая, следовательно существует нетривиальная линейная комбинация векторов этой системы равная нулевому вектору 
$\alpha_{1}a_{1} + \beta_{1}b_{1} + \dots + \beta_{n}b_{n} = 0_{v}$. Если $\beta_{i} = 0$, то $\alpha_{1}a_{1} = 0_{v}$. Так как система $A$ является базисом, то она линейно независима, следовательно её подсистема $A_{1}(a_{1})$ линейно независима $\Rightarrow$ линейная комбинация векторов системы $A_{1} = 0_{v}$ $\Rightarrow$ $\alpha_{1} = 0$ и следовательно исходная линейная комбинация тривиальная $\Rightarrow$ существует $\beta_{i} \neq 0$. Без ограничения общности считаем, что $\beta_{1} \neq 0$, иначе перенумеруем векторы $B$. Тогда $b_{1}$ линейно выражается через остальные векторы. $L(a_{1}, b_{1}, b_{2}, \dots, b_{n}) = L(a_{1}, b_{2}, \dots, b_{n}) = V$. Продолжая аналогичным образом получим, что $V = L(a_{1}, b_{2}, \dots, b_{n}) = L(a_{1}, a_{2}, b_{3}\dots,b_{n}) = \dots = L(a_{1}\dots,a_{n})$. Следовательно любой вектор $V$ линейно выражается через $(a_{1}\dots,a_{n})$ $\Rightarrow$ $A$ линейно зависимая. Противоречие с тем, что она базис $\Rightarrow$ $k = n$.
\end{proof}
\end{t2}


\newtheorem*{Definition4}{Определение}
\begin{Definition4}
Количество векторов в базисе конечномерного векторного пространства называются \textbf{размерностью} пространства(Обозначение: $dimV)$. Нулевое пространство принято считать нульмерным.
\end{Definition4}


\newtheorem*{t3}{Теорема}
\begin{t3}
Пусть $V-$ n-мерное векторное пространство. Тогда
\begin{enumerate}
\item система векторов, состоящая более чем из n векторов, линейно зависима;
\item линейно независимая система, состоящая из n векторов, является базисом;
\item линейно независимая система векторов, состоящая менее чем из n векторов, может быть дополнена до базиса.
\end{enumerate}
\begin{proof}
 Любая линейно независимая система векторов либо является базисом либо может быть дополнена до базиса, но в n-мерном пространстве все базисы состоят из n векторов.
Если система содержит n векторов то любое её пополнение будет линейно зависимым.
Система, состоящая менее чем из n векторов базисом не является, но может быть дополнена до базиса.
\end{proof}   
\end{t3}


\newtheorem*{Comment}{Замечание}
\begin{Comment}
Таким образом базис является максимальной линейно независимой системой векторов.
\end{Comment}




\end{document}