\documentclass[../../main.tex]{subfiles}
\begin{document}
\section{Канонический вид квадратичной формы}


\newtheorem*{CanonicalDef}{Определение}
\begin{CanonicalDef}
Квадратичная форма называется \underline{канонической}, если её матрица - диагональная,
то есть каноническая квадратичная форма имеет вид $\alpha_{1 1} x^{2}_{1}, \cdots, \alpha_{n n} x^{2}_{n}$.
\end{CanonicalDef}

\newtheorem*{CanonicalViewTheorem}{Теорема}
\begin{CanonicalViewTheorem}
Для любой квадратичной формы существует эквивалентная каноническая квадратичная форма.
\end{CanonicalViewTheorem}
\begin{proof}
Нереальное доказательство по индукции.
\end{proof}

\newtheorem*{CanonicalViewDef}{Определение}
\begin{CanonicalViewDef}
\underline{Каноническим видом} квадратичной формы называется эквивалентная ей каноническая форма.
\end{CanonicalViewDef}

\newtheorem*{AngleMinorsDef}{Определение}
\begin{AngleMinorsDef}
\underline{Угловыми минорами} матрицы $ A(\alpha_{i j}) $ называются миноры:
\[
\Delta_{1} =
\begin{vmatrix}
  \alpha_{1 1}
\end{vmatrix}
\text{, }
\Delta_{2} =
\begin{vmatrix}
  \alpha_{1 1} && \alpha_{1 2} \\
  \alpha_{2 1} && \alpha_{2 2}
\end{vmatrix}
\text{, }
\cdots
\text{, }
\Delta_{k} =
  \begin{vmatrix}
    \alpha_{1 1} && \cdots && \alpha_{1 k}    \\
    \vdots && \ddots && \vdots                \\
    \alpha_{k 1} && \cdots && \alpha_{k k}    \\
  \end{vmatrix}
\text{, }
\cdots
\text{, }
\Delta_{n} = \text{det A.}
\]
\end{AngleMinorsDef}

\newtheorem*{YakobyTh}{Теорема. (a.k.a. Формула Якоби)}
\begin{YakobyTh}
Если матрица квадратичной формы $f(x_{1}, \cdots, x_{n})$ имеет ненулевые угловые миноры, то
$f$ эквивалентна канонической квадратичной форме $\delta_{1} y_{1} + \cdots + \delta_{n} y^{2}_{n}$,
где $\delta_{i} = \frac{\Delta_{i - 1}}{\Delta{i}} (\Delta_{0} = 1)$.
\end{YakobyTh}

\end{document}
