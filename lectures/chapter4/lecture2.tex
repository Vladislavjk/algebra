\documentclass[../../main.tex]{subfiles}
\begin{document}
\section{Канонический вид квадратичной формы}


\newtheorem*{CanonicalDef}{Определение}
\begin{CanonicalDef}
Квадратичная форма называется \underline{канонической}, если её матрица - диагональная,
то есть каноническая квадратичная форма имеет вид $\alpha_{1 1} x^{2}_{1}, \cdots, \alpha_{n n} x^{2}_{n}$.
\end{CanonicalDef}

\newtheorem*{CanonicalViewTheorem}{Теорема}
\begin{CanonicalViewTheorem}
Для любой квадратичной формы существует эквивалентная каноническая квадратичная форма.
\end{CanonicalViewTheorem}
\begin{proof}
Индукцией по числу переменных.

\begin{itemize}
	
	\item При $n = 1$, любая квадратичная форма - каноническая.
	
	\item Пусть утвереждение теоремы верно для любой квадратичной формы с числом переменных меньше $n$, покажем, что оно верно и для $f(x_{1}, \dots, x_{n})$. Рассмотрим два случая:

	\textbf{Случай 1.} Среди диагональных коэффициентов $a_{i i}$ существует ненулевой. 

	Перенумеруем в квадратичной форме $f$ переменные так, чтобы отличным от нуля был коэффициент $a_{1 1}$. Перенумерация переменных -- линейное невырожденное преобразование, матрица которого мономиальна. Теперь имеем, что $a_{1 1} \ne 0$, тогда:
	\begin{equation}
		f(x_{1}, \dots, x_{n}) = a_{1 1} x^{2}_{1} + a_{1 2} x_{1} x_{2} + a_{2 1} x_{2} x_{1} + \dots + a_{1 n} x_{1} x_{n} + a_{n 1} x_{n} x_{1} + \sum^{n}_{i, j = 2} a_{i j} x_{i} x_{j}
	\end{equation}

	Теперь займемся выделением полного квадрата.

	\[
		(1) = \frac{1}{a_{1 1}} \left( a^{2}_{1 1} x^{2}_{1} + 2 a_{1 1} x_{1} (a_{1 2} x_{2} + \dots + a_{1 n} x_{n}) + (a_{1 2}x_{2} + \dots + a_{1 n} x_{n}) ^ {2}\right) - 
	\]
	\[
		- \frac{1}{a_{1 1}}\left( a_{1 2} x_{2} + \dots + a_{1 n} x_{n}\right)^{2} + \sum^{n}_{i, j = 2} a_{i j} x_{i} x_{j} = \frac{1}{a_{1 1}} \left( a_{1 1} x_{1} + a_{1 2} x_{2} + \dots + a_{1 n} x_{n} \right) ^{2} + \underline{\sum^{n}_{i, j = 2} a_{i j}x_{i} x_{j} - }
	\]
	
	\begin{equation}
		\underline{- \frac{1}{a_{1 1}}\left( a_{1 2} x_{2} + \dots + a_{1 n} x_{n}\right)}
	\end{equation}
	
	Обозначим подчеркнутую часть как $G(x_{2}, \dots, x_{n})$, заметим, что $G(x_{2}, \dots, x_{n})$ - квадратичная форма от $n - 1$ переменной, следовательно для неё существует эквивалентная каноническая форма (исходя из индуктивного предположения):
	\[
		\overline{G} (y_{2}, \dots, y_{n}) = \beta_{2 2} y^{2}_{2} + \dots + \beta_{n n} + y^{2}_{n}
	\]
	То есть существует линейное невырожденное преобразование переменных:
	\[
	\begin{cases}
		y_{2} = s_{2 2} x_{2} + \dots + s_{2 n} x_{n} \\
		\dots\dots\dots\dots\dots\dots\dots\dots\\
		y_{n} = s_{n 2} x_{2} + \dots + s_{n n} x_{n} \\
	\end{cases}
	\]
	переводящее квадратичную форму $\overline{G}$ в квадратичную форму $G$. Тогда линейное преобразование:
	\[
	\begin{cases}
		y_{1} = a_{1 1} x_{1} + \dots + a_{1 n} x_{n} \\
		y_{2} = s_{2 2} x_{2} + \dots + s_{2 n} x_{n} \\
		\dots\dots\dots\dots\dots\dots\dots\dots\\
		y_{n} = s_{n 2} x_{2} + \dots + s_{n n} x_{n} \\
	\end{cases}
	\]
	переводит каноническую квадратичную форму $\overline{G} (y_{2}, \dots, y_{n}) = \beta_{2 2} y^{2}_{2} + \dots + \beta_{n n} + y^{2}_{n}$ в квадратичную форму $f$. Причем определитель матрицы этого преобразования равен:
	\[
		\begin{vmatrix}
			a_{1 1} && a_{1 2} && \dots && a_{1 n} \\
			0 && s_{2 2} && \dots && s_{2 n} \\
			\vdots && \vdots &&  \dots && \vdots \\
			0 && s_{n 2} && \dots && s_{n n}
		\end{vmatrix}
		= a_{1 1} \times \text{det S} \ne 0\text{.}
	\]
	Так как матрица блочно-диагональная, то определитель равен произведению диагональных блоков, но исходя из того, что det $S$ $\ne 0$, так как матрица линейного преобразования $S$ невырожденная, и так как $a_{1 1} \ne 0$, то и весь определитель не равен 0. Отсюда следует, что $\overline{G} \sim f$

	\textbf{Случай 2.} Все диагональные коэффициенты $a_{i i}$ равны 0. 
	\begin{itemize}
		\itemЕсли $A = 0$, то $f$ уже каноническая.
		\itemЕсли $A \ne 0$, то существует $a_{i j} \ne 0$. Пусть, для определенности, этот коэффициент - $a_{1 2}$. Применим к квадратичной форме $f$ линейное невырожденное преобразование:
		\[
			\begin{cases} 
				x_{1} = y_{1} + y_{2}, \\
				x_{2} = y_{1} - y_{2}, \\
				x_{i} = y_{i}, & \forall i > 2;
			\end{cases}
		\]
		Матрица этого преобразования будет иметь следующий вид: 
		$
		\begin{pmatrix}
			1 & 1 & 0 & 0 & \vdots\\
			1 & -1 & 0 & 0 & \vdots \\
			0 & 0 & 1 & 0 & \vdots \\
			0 & 0 & 0 & 1 & \vdots \\
			\dots & \dots & \dots & \dots & \ddots
		\end{pmatrix}
		$. Легко понять, что определитель этой матрицы равен $-2$ -- значит это преобразование является невырожденным. В результате, вместо слагаемого $2 a_{1 2} x_{1} x_{2}$ получим слагаемое $2 a_{1 2} (y^{2}_{1} - y^{2}_{2})$ и, следовательно, коэффициент при $y^{2}_{1}$ равен $a_{1 2} \ne 0$, по доказанному выше, для такой квадратичной формы существует эквивалентная каноническая форма, и, учитывая транзитивность эквивалентности квадратичных форм, исходная квадратичная форма эквивалентна этой канонической форме.
	\end{itemize}
	

	
\end{itemize}


\end{proof}

\newtheorem*{CanonicalViewDef}{Определение}
\begin{CanonicalViewDef}
\underline{Каноническим видом} квадратичной формы называется эквивалентная ей каноническая форма.
\end{CanonicalViewDef}

\newtheorem*{AngleMinorsDef}{Определение}
\begin{AngleMinorsDef}
\underline{Угловыми минорами} матрицы $ A(\alpha_{i j}) $ называются миноры:
\[
\Delta_{1} =
\begin{vmatrix}
  \alpha_{1 1}
\end{vmatrix}
\text{, }
\Delta_{2} =
\begin{vmatrix}
  \alpha_{1 1} && \alpha_{1 2} \\
  \alpha_{2 1} && \alpha_{2 2}
\end{vmatrix}
\text{, }
\cdots
\text{, }
\Delta_{k} =
  \begin{vmatrix}
    \alpha_{1 1} && \cdots && \alpha_{1 k}    \\
    \vdots && \ddots && \vdots                \\
    \alpha_{k 1} && \cdots && \alpha_{k k}    \\
  \end{vmatrix}
\text{, }
\cdots
\text{, }
\Delta_{n} = \text{det A.}
\]
\end{AngleMinorsDef}

\newtheorem*{YakobyTh}{Теорема. (Формула Якоби)}
\begin{YakobyTh}
Если матрица квадратичной формы $f(x_{1}, \cdots, x_{n})$ имеет ненулевые угловые миноры, то
$f$ эквивалентна канонической квадратичной форме $\delta_{1} y_{1} + \cdots + \delta_{n} y^{2}_{n}$,
где $\delta_{i} = \frac{\Delta_{i - 1}}{\Delta{i}} (\Delta_{0} = 1)$.
\end{YakobyTh}

\end{document}
